\documentclass[
	12pt,
	oneside,
	a4paper,
	english,
	brazil,
]{abntex2}

% ---
% Pacotes fundamentais 
% ---
\usepackage{lmodern}
\usepackage[T1]{fontenc}
\usepackage[utf8]{inputenc}
\usepackage{indentfirst}
\usepackage{xcolor}
\usepackage{graphicx}
\usepackage{microtype}
\usepackage{listings}
\usepackage{multicol}
\usepackage{multirow}
\usepackage[brazilian,hyperpageref]{backref}
\usepackage[alf]{abntex2cite}

% ---
% Configuração do Listings para C#
% ---
\lstset{
  language=CSharp,
  captionpos=b,
  showspaces=false,
  showtabs=false,
  breaklines=true,
  showstringspaces=false,
  breakatwhitespace=true,
  commentstyle=\color{green!60!black},
  keywordstyle=\color{blue},
  stringstyle=\color{red},
  basicstyle=\ttfamily\small,
  frame=tb,
  tabsize=2,
    literate=
    *{á}{{\'a}}1 {é}{{\'e}}1 {í}{{\'i}}1 {ó}{{\'o}}1 {ú}{{\'u}}1
    {Á}{{\'A}}1 {É}{{\'E}}1 {Í}{{\'I}}1 {Ó}{{\'O}}1 {Ú}{{\'U}}1
    {à}{{\`a}}1 {è}{{\`e}}1 {ì}{{\`i}}1 {ò}{{\`o}}1 {ù}{{\`u}}1
    {ã}{{\~a}}1 {õ}{{\~o}}1 {Ã}{{\~A}}1 {Õ}{{\~O}}1
    {ç}{{\c c}}1 {Ç}{{\c C}}1
    {â}{{\^a}}1 {ê}{{\^e}}1 {î}{{\^i}}1 {ô}{{\^o}}1 {û}{{\^u}}1
}

% ---
% Informações de dados para CAPA e FOLHA DE ROSTO
% ---
\titulo{Prática 9 Laboratório de TAPOO}
\autor{Sergio Amaral de Souza}
\local{Belo Horizonte, Brasil}
\data{2024}
\instituicao{%
  Universidade Federal de Minas Gerais
  \par
  Colégio Técnico
  \par
  Curso Técnico em Desenvolvimento de Sistemas}

\definecolor{linkblue}{RGB}{41,5,195}

\makeatletter
\hypersetup{
	pdftitle={\@title},
	pdfauthor={\@author},
	pdfsubject={Relatório de Prática de Otimização},
	colorlinks=true,
	linkcolor=linkblue,
	citecolor=linkblue,
	filecolor=magenta,
	urlcolor=linkblue,
	bookmarksdepth=4
}
\makeatother

\renewcommand{\thesection}{\arabic{section}}
\setlength{\parindent}{1.3cm}
\setlength{\parskip}{0pt}

\makeindex

\begin{document}
\selectlanguage{brazil}
\frenchspacing
\imprimircapa
\textual

\section{Introdução}
\begin{itemize}
    \item Este laboratório aborda um problema de otimização de performance em um sistema de processamento de imagens em lote. A aplicação, que aplica um filtro de blur em centenas de imagens, sofria com alto consumo de memória e pausas frequentes do Garbage Collector (GC) devido a alocações excessivas.
    \item O objetivo do projeto foi refatorar a implementação ineficiente, substituindo a alocação contínua de memória pelo uso de \texttt{ArrayPool<T>}, um mecanismo de pooling de arrays de alta performance do .NET.
    \item A meta era demonstrar uma melhoria mensurável em tempo de execução, estabilidade e, principalmente, na redução da pressão sobre o GC, validando os resultados através de um benchmark comparativo entre a versão original e a otimizada.
\end{itemize}

\section{Desenvolvimento}

A solução foi desenvolvida em etapas: análise da implementação original, refatoração com a estratégia de pooling e, por fim, a criação de um sistema de benchmark para uma comparação quantitativa.

\subsection{Análise da Solução Trivial}
\begin{itemize}
    \item A versão original do processador de imagens alocava um novo array bidimensional para cada imagem a ser desfocada. O gargalo de performance residia no método \texttt{ApplyBlurFilter}, onde um novo array \texttt{blurred} era instanciado a cada chamada.
    \item Em um loop que processa 500 imagens, isso resultava em 500 alocações de grandes objetos na memória (LOH - Large Object Heap), sobrecarregando o Garbage Collector e causando coletas de lixo frequentes e custosas.
\end{itemize}

\begin{lstlisting}[caption={Ponto crítico de alocação na versão trivial.}]
private static PixelRGB[,] ApplyBlurFilter(PixelRGB[,] original)
{
    int height = original.GetLength(0);
    int width = original.GetLength(1);
    
    // PROBLEMA: Nova alocação a cada chamada, sobrecarregando o GC!
    var blurred = new PixelRGB[height, width];
    
    // ... Lógica de aplicação do filtro ...
    
    return blurred;
}
\end{lstlisting}

\subsection{Estratégia de Otimização com ArrayPool<T>}
\begin{itemize}
    \item Para resolver o problema, a estratégia foi substituir as alocações contínuas pelo uso de \texttt{ArrayPool<PixelRGB>.Shared}. Esta classe estática mantém um "pool" de arrays que podem ser "alugados" para uso temporário e "devolvidos" quando não são mais necessários.
    \item Isso evita o ciclo de criação e destruição de objetos, reduzindo drasticamente a pressão sobre o GC.
    \item Como \texttt{ArrayPool<T>} não suporta arrays multidimensionais nativamente, foi necessário simular as imagens 2D (800x600) com arrays unidimensionais de tamanho 480.000 (\texttt{width * height}).
\end{itemize}
A lógica do filtro foi então adaptada para usar o array 1D, calculando o índice correspondente a \texttt{[y, x]} com a fórmula \texttt{y * IMAGE\_WIDTH + x}.

\begin{lstlisting}[caption={Mapeamento de índice 2D para 1D no filtro otimizado.}]
private static void ApplyBlurFilter(PixelRGB[] original, PixelRGB[] blurred)
{
    for (int y = 0; y < IMAGE_HEIGHT - 1; y++)
    {
        for (int x = 0; x < IMAGE_WIDTH - 1; x++)
        {
            int currentIndex = y * IMAGE_WIDTH + x;
            // ... calcula os outros índices (direito, inferior, etc.)
            blurred[currentIndex] = PixelRGB.Average(/*...*/);
        }
    }
}
\end{lstlisting}

\subsection{Implementação Otimizada e Gerenciamento de Memória}
\begin{itemize}
    \item A versão otimizada gerencia o ciclo de vida dos arrays explicitamente. Dentro do loop, dois arrays são alugados do pool: um para a imagem original e outro para a imagem desfocada.
    \item Um bloco \texttt{try-finally} é essencial para garantir que os arrays sejam devolvidos ao pool com o método \texttt{Return()}, mesmo que ocorra uma exceção. Isso evita vazamentos de memória (memory leaks), onde a memória alugada nunca é liberada.
\end{itemize}

\begin{lstlisting}[caption={Ciclo de aluguel e devolução de arrays com ArrayPool.}]
var arrayPool = ArrayPool<PixelRGB>.Shared;
var bufferSize = IMAGE_WIDTH * IMAGE_HEIGHT;

for (int i = 0; i < TOTAL_IMAGES; i++)
{
    var originalImage = arrayPool.Rent(bufferSize);
    var blurredImage = arrayPool.Rent(bufferSize);

    try
    {
        // ... Lógica de processamento da imagem ...
    }
    finally
    {
        arrayPool.Return(originalImage);
        arrayPool.Return(blurredImage);
    }
}
\end{lstlisting}

\subsection{Metodologia de Benchmark}
\begin{itemize}
    \item Para validar a otimização, um programa de benchmark foi criado para executar ambas as versões em sequência e coletar métricas de performance de forma padronizada.
    \item O tempo de execução foi medido com a classe \texttt{System.Diagnostics.Stopwatch}.
    \item As coletas de lixo e a memória alocada foram monitoradas com a classe \texttt{System.GC}, capturando os contadores de cada geração (Gen 0, 1 e 2) e a memória total antes e depois de cada execução.
\end{itemize}

\begin{lstlisting}[caption={Coleta de métricas de GC para análise.}]
// Captura do estado inicial
long initialMemory = GC.GetTotalMemory(true);
var gen0Before = GC.CollectionCount(0);
var gen1Before = GC.CollectionCount(1);
var gen2Before = GC.CollectionCount(2);

// ... Execução do processamento de imagens ...

// Captura do estado final e cálculo da diferença
long finalMemory = GC.GetTotalMemory(true);
var gc0Count = GC.CollectionCount(0) - gen0Before;
var gc1Count = GC.CollectionCount(1) - gen1Before;
var gc2Count = GC.CollectionCount(2) - gen2Before;
\end{lstlisting}

\section{Resultados}

A execução do benchmark forneceu uma comparação clara entre as duas abordagens. A imagem a seguir exibe a tabela de resultados gerada pelo programa, que resume as métricas de performance coletadas para cada versão.

\begin{figure}[h!]
  \centering
  \includegraphics[width=\linewidth]{ap.png}
  \caption{Saída do terminal com a tabela de resultados do benchmark comparativo.}
  \label{fig:resultados}
\end{figure}

A análise detalhada da tabela revela o seguinte:
\begin{itemize}
    \item \textbf{Tempo Total (ms):} A versão otimizada foi consistentemente mais rápida, com uma melhoria de aproximadamente \textbf{21\%}. Essa aceleração é um resultado direto da redução drástica do trabalho do Garbage Collector. Menos tempo gasto em coletas de lixo significa mais tempo de CPU disponível para o processamento real das imagens.
    \item \textbf{Alocação de Memória (MB):} Esta métrica pode ser contraintuitiva. O valor maior na versão otimizada não indica um problema, mas sim o funcionamento do \texttt{ArrayPool}, que pré-aloca e retém memória para reutilização. \texttt{GC.GetTotalMemory()} mede a memória total em uso, que inclui esse pool. O verdadeiro ganho, que é a ausência de alocações \textit{constantes}, é melhor observado na próxima métrica.
    \item \textbf{Coleções GC (Gen 0, 1 e 2):} Este é o resultado mais impactante. A versão trivial executou 250 coletas de lixo, enquanto a otimizada necessitou de apenas 2. Isso representa uma redução de \textbf{99.2\%} e prova o sucesso da otimização. Menos coletas significam uma execução mais estável, previsível e com menos pausas ("engasgos").
\end{itemize}

\clearpage
\section{Conclusão}

\begin{itemize}
    \item \textbf{Resultados vs. Expectativas:} O experimento foi bem-sucedido e os resultados alcançaram os objetivos propostos. A aplicação da técnica de pooling com \texttt{ArrayPool<T>} demonstrou ser extremamente eficaz para o cenário de alocação repetitiva de grandes objetos, validando a teoria na prática.
    \item \textbf{Dificuldades Encontradas:} A principal dificuldade não foi técnica, mas sim conceitual: a interpretação correta das métricas de memória. Foi necessário compreender que, em um cenário de pooling, um maior uso de memória reportado pelo GC ao final da execução é esperado e não indica um problema, mas sim a eficácia da estratégia de reuso.
    \item \textbf{Aprendizado:} O principal aprendizado foi a confirmação prática da importância da gestão consciente da memória em aplicações .NET de alta performance. Reduzir a "pressão" sobre o Garbage Collector é uma das otimizações mais eficazes, resultando em ganhos expressivos de desempenho e estabilidade, muitas vezes sem alterar a lógica de negócio fundamental do algoritmo.
    \item \textbf{Melhorias Futuras:} O projeto poderia ser estendido para explorar o uso de \texttt{Memory<T>} e \texttt{Span<T>}, que trabalham em conjunto com o \texttt{ArrayPool} para fornecer uma abstração ainda mais segura e eficiente sobre a memória, evitando cópias de dados desnecessárias.
\end{itemize}

\end{document}
