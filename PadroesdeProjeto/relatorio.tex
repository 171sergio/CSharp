\documentclass[
    12pt,
    oneside,
    a4paper,
    english,
    brazil,
]{abntex2}

% Pacotes fundamentais
\usepackage{lmodern}
\usepackage[T1]{fontenc}
\usepackage[utf8]{inputenc}
\usepackage{indentfirst}
\usepackage{color}
\usepackage{graphicx}
\usepackage{microtype}
\usepackage{listings}
\usepackage{multicol}
\usepackage{multirow}
\usepackage[brazilian,hyperpageref]{backref}
\usepackage[alf]{abntex2cite}

\lstset{
  language=CSharp,
  showspaces=false,
  showtabs=false,
  breaklines=true,
  showstringspaces=false,
  breakatwhitespace=true,
  commentstyle=\color{green!60!black},
  keywordstyle=\color{blue},
  stringstyle=\color{red},
  basicstyle=\ttfamily\small,
  frame=tb,
  tabsize=2
}

% Informações de dados para CAPA e FOLHA DE ROSTO
\titulo{Sistema de E-commerce com Padrões de Projeto}
\autor{Seu Nome}
\local{Belo Horizonte, Brasil}
\data{2024}
\instituicao{
  Universidade Federal de Minas Gerais
  \par
  Colégio Técnico
  \par
  Curso Técnico em Desenvolvimento de Sistemas}

\definecolor{linkblue}{RGB}{41,5,195}

\makeatletter
\hypersetup{
    pdftitle={\@title},
    pdfauthor={\@author},
    pdfsubject={Relatório de Prática de Padrões de Projeto},
    colorlinks=true,
    linkcolor=linkblue,
    citecolor=linkblue,
    filecolor=magenta,
    urlcolor=linkblue,
    bookmarksdepth=4
}
\makeatother

\renewcommand{\thesection}{\arabic{section}}
\setlength{\parindent}{1.3cm}
\setlength{\parskip}{0pt}

\makeindex

\begin{document}
\selectlanguage{brazil}
\frenchspacing
\imprimircapa
\textual

\section{Introdução}
\begin{itemize}
    \item Este relatório apresenta o desenvolvimento de um sistema de e-commerce simplificado em C\#, projetado para demonstrar a aplicação de cinco padrões de projeto: Singleton, Factory Method, Observer, Strategy e Decorator.
    \item O sistema gerencia produtos de diferentes categorias, pedidos, métodos de pagamento e notificações, promovendo flexibilidade e extensibilidade.
    \item O objetivo é evidenciar como padrões de projeto podem ser aplicados para resolver problemas comuns de arquitetura e design em sistemas reais.
\end{itemize}

\section{Desenvolvimento}
O sistema foi estruturado em torno dos seguintes padrões:

\subsection{Singleton: Gerenciador de Configurações}
\begin{lstlisting}[caption={Singleton para configurações globais}]
public sealed class GerenciadorConfiguracao {
    private static GerenciadorConfiguracao _instancia;
    private static readonly object _bloqueio = new object();
    private GerenciadorConfiguracao() { }
    public static GerenciadorConfiguracao Instancia {
        get {
            if (_instancia == null) {
                lock (_bloqueio) {
                    if (_instancia == null)
                        _instancia = new GerenciadorConfiguracao();
                }
            }
            return _instancia;
        }
    }
    public string ConexaoBancoDados { get; set; } = "ConexaoPadrao";
    public decimal TaxaImposto { get; set; } = 0.08m;
}
\end{lstlisting}

\subsection{Factory Method: Criação de Produtos}
\begin{lstlisting}[caption={Factory para produtos}]
public abstract class FabricaProduto {
    public abstract Produto CriarProduto(string nome, decimal preco);
}
public class FabricaEletronicos : FabricaProduto {
    public override Produto CriarProduto(string nome, decimal preco) {
        return new Eletronico { Nome = nome, Preco = preco };
    }
}
// Implementações similares para Roupa e Livro
\end{lstlisting}

\subsection{Observer: Sistema de Notificações}
\begin{lstlisting}[caption={Observer para notificações de pedido}]
public interface IObservadorPedido {
    void AoMudarStatusPedido(Pedido pedido, string novoStatus);
}
public class Pedido {
    private List<IObservadorPedido> _observadores = new List<IObservadorPedido>();
    private string _status;
    public string Status {
        get => _status;
        set {
            _status = value;
            NotificarObservadores();
        }
    }
    public void Inscrever(IObservadorPedido observador) {
        _observadores.Add(observador);
    }
    private void NotificarObservadores() {
        foreach (var observador in _observadores) {
            observador.AoMudarStatusPedido(this, _status);
        }
    }
}
\end{lstlisting}

\subsection{Strategy: Métodos de Pagamento}
\begin{lstlisting}[caption={Strategy para métodos de pagamento}]
public interface IEstrategiaPagamento {
    bool ProcessarPagamento(decimal valor);
    string ObterDetalhespagamento();
}
public class ContextoPagamento {
    private IEstrategiaPagamento _estrategiaPagamento;
    public void DefinirEstrategiaPagamento(IEstrategiaPagamento estrategia) {
        _estrategiaPagamento = estrategia;
    }
    public bool ExecutarPagamento(decimal valor) {
        return _estrategiaPagamento?.ProcessarPagamento(valor) ?? false;
    }
}
\end{lstlisting}

\subsection{Decorator: Funcionalidades Extras do Produto}
\begin{lstlisting}[caption={Decorator para produtos}]
public abstract class DecoradorProduto : Produto {
    protected Produto _produto;
    public DecoradorProduto(Produto produto) {
        _produto = produto;
    }
    public override string ObterCategoria() => _produto.ObterCategoria();
    public override decimal CalcularFrete() => _produto.CalcularFrete();
}
public class DecoradorGarantia : DecoradorProduto {
    private int _mesesGarantia;
    public DecoradorGarantia(Produto produto, int mesesGarantia) : base(produto) {
        _mesesGarantia = mesesGarantia;
        Nome = produto.Nome;
        Preco = produto.Preco + (mesesGarantia * 10);
    }
    public override string ObterCategoria() => _produto.ObterCategoria() + " + Garantia";
}
\end{lstlisting}

\section{Resultados}
A seguir, apresentamos a execução do sistema, demonstrando a integração dos padrões e o funcionamento do e-commerce.\newline
\begin{figure}[h!]
  \centering
  \includegraphics[width=0.9\linewidth]{resultado.png}
  \caption{Saída do terminal com a execução do sistema de e-commerce.}
  \label{fig:resultados}
\end{figure}

\section{Conclusão}
\begin{itemize}
    \item O sistema desenvolvido demonstra, de forma prática, a aplicação de cinco padrões de projeto fundamentais em um contexto real de e-commerce.
    \item A utilização dos padrões Singleton, Factory Method, Observer, Strategy e Decorator proporcionou um código mais modular, flexível e de fácil manutenção.
    \item O exercício evidenciou a importância de boas práticas de design para a escalabilidade e clareza de sistemas orientados a objetos.
    \item Como melhoria futura, seria interessante adicionar testes automatizados e expandir o sistema para suportar mais tipos de produtos e integrações reais de pagamento.
\end{itemize}

\end{document} 