
\documentclass[
	12pt,				% tamanho da fonte
	oneside,			% para impressão em recto e verso. Oposto a oneside
	a4paper,			% tamanho do papel. 
	english,			% idioma adicional para hifenização
	brazil,				% o último idioma é o principal do documento
	]{abntex2}

% ---
% Pacotes fundamentais 
% ---
\usepackage{lmodern}			% Usa a fonte Latin Modern
\usepackage[T1]{fontenc}		% Selecao de codigos de fonte.
\usepackage[utf8]{inputenc}		% Codificacao do documento (conversão automática dos acentos)
\usepackage{indentfirst}		% Indenta o primeiro parágrafo de cada seção.
\usepackage{color}				% Controle das cores
\usepackage{graphicx}			% Inclusão de gráficos
\usepackage{microtype} 			% para melhorias de justificação
\usepackage{multicol}
\usepackage{multirow}
\usepackage[brazilian,hyperpageref]{backref}	 % Paginas com as citações na bibl
\usepackage[alf]{abntex2cite}	% Citações padrão ABNT
\usepackage{listings}

\lstset{language=Java,
  showspaces=false,
  showtabs=false,
  breaklines=true,
  showstringspaces=false,
  breakatwhitespace=true,
  commentstyle=\color{green},
  keywordstyle=\color{blue},
  stringstyle=\color{red},
  basicstyle=\ttfamily
}

% --- 
% CONFIGURAÇÕES DE PACOTES
% --- 

% ---
% Configurações do pacote backref
% Usado sem a opção hyperpageref de backref
\renewcommand{\backrefpagesname}{Citado na(s) página(s):~}
% Texto padrão antes do número das páginas
\renewcommand{\backref}{}
% Define os textos da citação
\renewcommand*{\backrefalt}[4]{
	\ifcase #1 %
		Nenhuma citação no texto.%
	\or
		Citado na página #2.%
	\else
		Citado #1 vezes nas páginas #2.%
	\fi}%
% ---

% ---
% Informações de dados para CAPA e FOLHA DE ROSTO
% ---
\titulo{Prática XXX\\Laboratório de YYY}
\autor{Aluno1, Aluno2, Aluno3}
\local{Belo Horizonte, Brasil}
\data{202X}
\instituicao{%
  Universidade Federal de Minas Gerais
  \par
  Colégio Técnico
  \par
  Curso Técnico em XXX}

\definecolor{blue}{RGB}{41,5,195}

\makeatletter
\hypersetup{
     	%pagebackref=true,
		pdftitle={\@title}, 
		pdfauthor={\@author},
    	pdfsubject={\imprimirpreambulo},
		colorlinks=true,       		% false: boxed links; true: colored links
    	linkcolor=blue,          	% color of internal links
    	citecolor=blue,        		% color of links to bibliography
    	filecolor=magenta,      		% color of file links
		urlcolor=blue,
		bookmarksdepth=4
}
\makeatother

\renewcommand{\thesection}{\arabic{section}}
\setlength{\parindent}{1.3cm}
\setlength{\parskip}{0.2cm} 

\makeindex


\begin{document}

\selectlanguage{brazil}
\frenchspacing 

\imprimircapa

{
\ABNTEXchapterfont

\textual

% ----------------------------------------------------------
% Introdução (exemplo de capítulo sem numeração, mas presente no Sumário)
% ----------------------------------------------------------
\section{Introdução}

Nesta seção, você deve:
\begin{itemize}
    \item Apresentar uma descrição detalhada do problema utilizado no laboratório, incluindo suas características principais e seu funcionamento geral.
    \item Especificar os objetivos do experimento, detalhando o que se esperava aprender ou demonstrar com a prática.
\end{itemize}

\section{Desenvolvimento}

Aqui, detalhe:
\begin{itemize}
    \item A estratégia adotada para solucionar o problema, incluindo a seleção de métodos e técnicas.
    \item Os passos seguidos durante a execução do experimento, possivelmente incluindo configurações de parâmetros, design de soluções, e máquinas de estados.
    \item Qualquer fundamentação teórica que suporte as escolhas feitas durante o desenvolvimento do projeto.
\end{itemize}

\begin{lstlisting}
int[][] grid;
int n = 100;

void setup(){
    size(600,600);
    frameRate(60);
    grid = criaGrid();
}
\end{lstlisting}

\section{Resultados}

Nesta parte, descreva:
\begin{itemize}
    \item Os resultados obtidos com a implementação da estratégia de controle. Inclua dados quantitativos e qualitativos, como gráficos, tabelas e observações relevantes, código desenvolvido.
    \item Uma análise crítica dos resultados, comparando-os com os objetivos propostos na Introdução. Discuta qualquer desvio ou correspondência notável.
\end{itemize}

\section{Conclusão}

Por fim, sumarize:
\begin{itemize}
    \item O quanto os resultados alcançados estão de acordo com o esperado, baseando-se na teoria e nos objetivos do experimento.
    \item As principais dificuldades encontradas durante a execução do experimento e como foram superadas (ou não).
    \item O que foi aprendido com a realização do experimento, destacando o valor prático e teórico da experiência.
\end{itemize}



\end{document}
